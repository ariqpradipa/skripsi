%
% Halaman Abstrak
%
% @author  Andreas Febrian
% @version 1.00
%

\chapter*{Abstrak}

\vspace*{0.2cm}
{
	\setlength{\parindent}{0pt}

	\begin{tabular}{@{}l l p{10cm}}
		Nama          & : & \penulis    \\
		Program Studi & : & \program    \\
		Judul         & : & \judul      \\
		Pembimbing    & : & \pembimbing \\
	\end{tabular}

	\bigskip
	\bigskip

	\cc\ menyediakan akses mudah ke berbagai sumber daya melalui jaringan internet, dengan jangkauan jaringan yang luas, kemampuan elastisitas yang cepat, dan layanan yang dapat diukur. Dalam penelitian ini, digunakan \f{Cloud Management Platform} Apache CloudStack untuk membangun sistem Cloud sendiri. Sistem ini menggunakan sistem operasi Ubuntu dan hypervisor KVM (Kernel-based Virtual Machine) untuk menguji performa dan optimalisasi konfigurasi hypervisor dalam menjalankan tugas-tugas komputasi tertentu.

	Penelitian ini bertujuan untuk menganalisis apakah konfigurasi default dari hypervisor KVM sudah optimal dan bagaimana konfigurasi hypervisor KVM dapat meningkatkan performa dalam menjalankan tugas komputasi seperti kompresi video, enkripsi, dekripsi, dan validasi integritas data. Metode yang digunakan melibatkan pengujian performa hypervisor KVM dengan melakukan berbagai tugas komputasi dan mengukur waktu yang diperlukan untuk menyelesaikan tugas-tugas tersebut. Hasil pengujian menunjukkan bahwa konfigurasi hypervisor dengan menambahkan flag tertentu seperti SSSE3, SSE4.1, SSE4.2, SSE4a, dan AES dapat meningkatkan performa komputasi secara signifikan.

	Kesimpulan dari penelitian ini adalah bahwa konfigurasi default hypervisor KVM bersifat minimal dan perlu dilakukan konfigurasi untuk mencapai performa optimal. Konfigurasi hypervisor yang tepat dapat mempercepat kompresi video hingga 2.4 kali, validasi integritas data hingga 7.56 kali, enkripsi AES hingga 2.1 kali, dan dekripsi AES hingga 2.68 kali dibandingkan dengan konfigurasi default. Penelitian ini diharapkan memberikan kontribusi akademis dalam bidang optimalisasi infrastruktur cloud.

	\bigskip

	Kata kunci:\\
	Cloud Computing, KVM Hypervisor, Optimisasi Performa, Apache CloudStack
}

\newpage