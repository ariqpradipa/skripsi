%
% Halaman Abstrak
%
% @author  Andreas Febrian
% @version 1.00
%

\chapter*{Abstrak}

\vspace*{0.2cm}
{
	\setlength{\parindent}{0pt}
	
	\begin{tabular}{@{}l l p{10cm}}
		Nama&: & \penulis \\
		Program Studi&: & \program \\
		Judul&: & \judul \\
		Pembimbing&: & \pembimbing \\
	\end{tabular}

	\bigskip
	\bigskip
	
	Penelitian ini bertujuan untuk menganalisis performa dari tuning Hypervisor Kernel-based Virtual Machine (KVM) dengan fokus pada Streaming SIMD Extensions (SSE) dalam konteks kompresi video pada Virtual Machines (VMs) yang berjalan di Apache CloudStack. Kajian ini secara khusus meneliti pengaruh penambahan flag SSE3, SSE4.1, dan SSE4.2 terhadap performa kompresi video. Dua aspek utama yang menjadi fokus adalah kecepatan kompresi dan kualitas video yang dihasilkan.
	
	Untuk menilai kecepatan kompresi, penelitian ini membandingkan waktu kompresi antara Hypervisor KVM yang telah di-tuning dengan flag SSE tambahan dan Hypervisor KVM default. Metode yang digunakan untuk mengukur kecepatan ini adalah dengan menghitung waktu yang diperlukan untuk kompresi video pada kedua konfigurasi hypervisor tersebut.
	
	Di sisi lain, kualitas video yang dikompresi dievaluasi menggunakan metrik Structural Similarity Index Measure (SSIM). SSIM digunakan untuk mengukur kesamaan visual antara video asli dan video hasil kompresi, memberikan gambaran tentang seberapa baik video dikompresi tanpa kehilangan kualitas visual.
	
	Penelitian ini menghasilkan temuan penting mengenai efektivitas tuning SSE pada Hypervisor KVM dalam konteks Apache CloudStack, memberikan wawasan baru tentang optimisasi performa untuk kompresi video di lingkungan cloud. Hasil ini diharapkan dapat membantu dalam pengembangan praktik terbaik untuk konfigurasi Hypervisor KVM dalam meningkatkan efisiensi dan efektivitas kompresi video pada VMs.

	\bigskip

	Kata kunci:\\
	Tuning Hypervisor KVM, Kompresi Video, Cloud Computing, Apache CloudStack
}

\newpage