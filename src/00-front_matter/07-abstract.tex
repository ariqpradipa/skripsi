%
% Halaman Abstract
%
% @author  Andreas Febrian
% @version 1.00
%

\chapter*{Abstract}

\vspace*{0.2cm}
{
	\setlength{\parindent}{0pt}
	
	\begin{tabular}{@{}l l p{10cm}}
		Name&: & \penulis \\
		Study Program&: & \program \\
		Title&: & \judulInggris \\
		Counsellor&: & \pembimbing \\
	\end{tabular}

	\bigskip
	\bigskip

	Cloud Computing provides easy access to various resources over the internet, with a wide network reach, fast elasticity capabilities, and measurable services. In this research, the Apache CloudStack Cloud Management Platform is utilized to build a Cloud system. This system uses the Ubuntu operating system and KVM (Kernel-based Virtual Machine) hypervisor to test performance and optimize the hypervisor configuration for specific computing tasks.
	
	The aim of this research is to analyze whether the default configuration of the KVM hypervisor is optimal and how tuning the KVM hypervisor can enhance performance in executing computing tasks such as video compression, encryption, decryption, and data integrity validation. The methodology involves performance testing of the KVM hypervisor by executing various computing tasks and measuring the time taken to complete these tasks. The test results indicate that tuning the hypervisor by adding specific flags such as SSSE3, SSE4.1, SSE4.2, SSE4a, and AES can significantly improve computational performance.
	
	The conclusion drawn from this research is that the default configuration of the KVM hypervisor is minimal and requires tuning to achieve optimal performance. Proper hypervisor tuning can accelerate video compression by up to 2.4 times, data integrity validation by up to 7.56 times, AES encryption by up to 2.1 times, and AES decryption by up to 2.68 times compared to the default configuration. This research is expected to make an academic contribution in the field of cloud infrastructure optimization.

	\bigskip

	Key words:\\
	Cloud Computing, KVM Hypervisor, Performance Optimization, Apache CloudStack
}

\newpage