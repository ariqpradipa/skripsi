%
% Halaman Abstract
%
% @author  Andreas Febrian
% @version 1.00
%

\chapter*{Abstract}

\vspace*{0.2cm}
{
	\setlength{\parindent}{0pt}
	
	\begin{tabular}{@{}l l p{10cm}}
		Name&: & \penulis \\
		Study Program&: & \program \\
		Title&: & \judulInggris \\
		Counsellor&: & \pembimbing \\
	\end{tabular}

	\bigskip
	\bigskip

	This study aims to analyze the performance of tuning the Kernel-based Virtual Machine (KVM) Hypervisor, focusing on Streaming SIMD Extensions (SSE) in the context of video compression on Virtual Machines (VMs) running in Apache CloudStack. Specifically, this research examines the impact of adding SSE3, SSE4.1, and SSE4.2 flags on video compression performance. The main focuses are the speed of compression and the quality of the resulting video.
	
	To assess compression speed, this study compares the compression time between the KVM Hypervisor tuned with additional SSE flags and the default KVM Hypervisor. The method used to measure this speed involves calculating the time required for video compression on both hypervisor configurations.
	
	On the other hand, the quality of the compressed video is evaluated using the Structural Similarity Index Measure (SSIM) metric. SSIM is used to measure the visual similarity between the original video and the compressed video, providing an insight into how well the video is compressed without losing visual quality.
	
	This research yields significant findings regarding the effectiveness of SSE tuning on the KVM Hypervisor in the context of Apache CloudStack, offering new insights into performance optimization for video compression in cloud environments. These results are expected to assist in developing best practices for configuring the KVM Hypervisor to enhance efficiency and effectiveness in video compression on VMs.

	\bigskip

	Key words:\\
	KVM Hypervisor Tuning, Video Compression, Cloud Computing, Apache CloudStack
}

\newpage