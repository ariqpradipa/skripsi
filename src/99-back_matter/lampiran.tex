%-----------------------------------------------------------------------------%
\addChapter{Lampiran 1}
\chapter*{Lampiran 1}
%-----------------------------------------------------------------------------%

\begin{listing}[H]
    \begin{minted}{bash}
    #!/bin/bash

    # Specify the path to your input video file
    input_file="timessquaredaywide.mp4"

    # Specify the output directory
    output_directory="output"

    # Specify the number of times to run the benchmark
    num_runs=100

    # Initialize total time to zero
    total_time=0

    # Loop to run the benchmark command multiple times
    for i in $(seq 1 $num_runs)
    do
        start_time=$(date +%s.%N)
        log_file="$output_directory/log_$i.txt"
        
        echo "Running benchmark $i..."
        
        # Add your ffmpeg compression command here
        ffmpeg -i "$input_file" -c:v libx264 -preset medium -crf 32 -c:a aac -strict -2 "$output_directory/output_$i.mp4" > "$log_file" 2>&1
        
        end_time=$(date +%s.%N)
        elapsed_time=$(awk "BEGIN {print $end_time - $start_time}")
        
        echo "Benchmark $i completed in $elapsed_time seconds. Log saved to $log_file."
        
        # Accumulate the elapsed time
        total_time=$(awk "BEGIN {print $total_time + $elapsed_time}")
    done

    # Calculate the average time
    average_time=$(awk "BEGIN {print $total_time / $num_runs}")

    # Print the total average time
    echo "Total average time for $num_runs benchmarks: $average_time seconds."
    \end{minted}
    \caption{Kode Pengujian Kompresi Video}
    \label{code:kode_pengujian_kompresi_video}
\end{listing}

\begin{listing}[H]
    \begin{minted}{python}
    import os
    import time
    import crc32c


    def benchmark_crc32c(data, iterations):
        # Perform CRC32 calculation
        start_time = time.time()
        for _ in range(iterations):
            crc32c.crc32c(data)
        end_time = time.time()

        # Calculate average time per iteration
        avg_time = (end_time - start_time) / iterations
        return avg_time


    def main():
        # Define parameters
        data_size = 10 * 1024 * 1024  # Size of the data (10 MB)
        num_iterations = 100  # Number of iterations to average the time

        # Generate random data
        data = bytearray(os.urandom(data_size))

        # Run the benchmark
        avg_time_crc32c = benchmark_crc32c(data, num_iterations)
        print(
            f"Average time for CRC32 calculation: {avg_time_crc32c * 1000} ms"
        )  # convert to ms


    main()
    \end{minted}
    \caption{Kode Pengujian Validasi Integritas Data}
    \label{code:kode_pengujian_validasi_integritas_data}
\end{listing}

\begin{listing}
    \begin{minted}{python}
    from cryptography.hazmat.primitives.ciphers import Cipher, algorithms, modes
    from cryptography.hazmat.primitives import padding
    from cryptography.hazmat.backends import default_backend
    import time

    # Generate a random AES key and plaintext
    key = b"sixteenbytekey12"
    plaintext = b"The Advanced Encryption Standard (AES) is a symmetric-key encryption algorithm that has become the de facto standard for securing sensitive data worldwide. Developed by Belgian cryptographers Joan Daemen and Vincent Rijmen, AES was adopted by the U.S. National Institute of Standards and Technology (NIST) in 2001 after a rigorous evaluation process involving numerous submissions from around the globe. AES replaced the aging Data Encryption Standard (DES) and its variants, which were becoming increasingly vulnerable to brute-force attacks due to advances in computing power. AES operates on fixed block sizes of 128 bits and supports key lengths of 128, 192, or 256 bits, with the latter providing the highest level of security. Its sophisticated mathematical structure, based on substitution-permutation networks, makes it highly resistant to various cryptanalytic attacks, including differential and linear cryptanalysis. AES employs multiple rounds of substitution, permutation, and key mixing operations to scramble the plaintext data, making it virtually impossible to decrypt without the correct key. One of the key strengths of AES is its versatility and flexibility. It can be implemented in hardware or software, making it suitable for a wide range of applications, from securing communication channels to protecting data at rest. AES has been extensively studied and scrutinized by cryptographers worldwide, and its strength lies in its ability to withstand known attacks while maintaining high performance. Consequently, AES has been widely adopted in various industries, including government, finance, healthcare, and telecommunications, as well as in protocols and standards such as TLS/SSL, IPsec, and wireless encryption standards like WPA2."


    # Function to perform AES encryption benchmark
    def aes_encrypt_benchmark(key, plaintext):
        backend = default_backend()
        cipher = Cipher(algorithms.AES(key), modes.ECB(), backend=backend)
        encryptor = cipher.encryptor()

        padder = padding.PKCS7(algorithms.AES.block_size).padder()
        padded_plaintext = padder.update(plaintext) + padder.finalize()

        start_time = time.time()
        ciphertext = encryptor.update(padded_plaintext) + encryptor.finalize()
        end_time = time.time()

        return ciphertext, end_time - start_time


    # Function to perform AES decryption benchmark
    def aes_decrypt_benchmark(key, ciphertext):
        backend = default_backend()
        cipher = Cipher(algorithms.AES(key), modes.ECB(), backend=backend)
        decryptor = cipher.decryptor()

        start_time = time.time()
        padded_plaintext = decryptor.update(ciphertext) + decryptor.finalize()
        end_time = time.time()

        unpadder = padding.PKCS7(algorithms.AES.block_size).unpadder()
        plaintext = unpadder.update(padded_plaintext) + unpadder.finalize()

        return end_time - start_time
    \end{minted}
\end{listing}
\begin{listing}
    \begin{minted}{python}
    def run_benchmark():
        # Benchmark AES encryption
        ciphertext, aes_time_encrypt = aes_encrypt_benchmark(key, plaintext)
        # print(f"AES Encryption Time: {(aes_time_encrypt * 1000):.5f} ms")

        # Benchmark AES decryption
        aes_time_decrypt = aes_decrypt_benchmark(key, ciphertext)
        # print(f"AES Decryption Time: {(aes_time_decrypt * 1000):.5f} ms")
        return aes_time_encrypt, aes_time_decrypt


    def main(i):
        iterations = 100

        encrypt_time = 0
        decrypt_time = 0
        for _ in range(iterations):
            encrypt_time_iter, decrypt_time_iter = run_benchmark()
            encrypt_time += encrypt_time_iter
            decrypt_time += decrypt_time_iter

        avg_encrypt_time = encrypt_time / iterations
        avg_decrypt_time = decrypt_time / iterations

        print(f"Run {i+1} Average AES Encryption Time: {(avg_encrypt_time * 1000):.5f} ms")
        print(
            f"Run {i+1} Average AES Decryption Time: {(avg_decrypt_time * 1000):.5f} ms\n"
        )
        
    for i in range(1):
        main(i)
    \end{minted}
    \caption{Kode Pengujian Enkripsi AES}
    \label{code:pengujian_enkripsi_aes}
\end{listing}


\begin{figure}
    \begin{quote}
        The Advanced Encryption Standard (AES) is a symmetric-key encryption algorithm that has become the de facto standard for securing sensitive data worldwide. Developed by Belgian cryptographers Joan Daemen and Vincent Rijmen, AES was adopted by the U.S. National Institute of Standards and Technology (NIST) in 2001 after a rigorous evaluation process involving numerous submissions from around the globe. AES replaced the aging Data Encryption Standard (DES) and its variants, which were becoming increasingly vulnerable to brute-force attacks due to advances in computing power. AES operates on fixed block sizes of 128 bits and supports key lengths of 128, 192, or 256 bits, with the latter providing the highest level of security. Its sophisticated mathematical structure, based on substitution-permutation networks, makes it highly resistant to various cryptanalytic attacks, including differential and linear cryptanalysis. AES employs multiple rounds of substitution, permutation, and key mixing operations to scramble the plaintext data, making it virtually impossible to decrypt without the correct key. One of the key strengths of AES is its versatility and flexibility. It can be implemented in hardware or software, making it suitable for a wide range of applications, from securing communication channels to protecting data at rest. AES has been extensively studied and scrutinized by cryptographers worldwide, and its strength lies in its ability to withstand known attacks while maintaining high performance. Consequently, AES has been widely adopted in various industries, including government, finance, healthcare, and telecommunications, as well as in protocols and standards such as TLS/SSL, IPsec, and wireless encryption standards like WPA2.
    \end{quote}
    \caption{Plaintext Data untuk AES Encryption Benchmark}
    \label{fig:aesPlaintextData}
\end{figure}