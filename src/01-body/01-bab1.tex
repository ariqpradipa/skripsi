%-----------------------------------------------------------------------------%
\chapter{\babSatu}
%-----------------------------------------------------------------------------%
% \todo{tambahkan kata-kata pengantar bab 1 disini}


%-----------------------------------------------------------------------------%
\section{Latar Belakang}
%-----------------------------------------------------------------------------%
\cc\ adalah sebuah model yang memungkinkan akses yang luas, nyaman, dan sesuai permintaan ke sumber daya komputasi bersama, seperti jaringan, server, dan penyimpanan. Model ini terdiri dari lima karakteristik utama, yakni layanan mandiri \f{on-demand}, akses jaringan yang luas, pengelolaan sumber daya \f{(resource pooling)}, elastisitas yang cepat, dan layanan yang terukur. \cc\ juga melibatkan tiga model layanan, seperti \f{Software as a Service} (SaaS), \f{Platform as a Service} (PaaS), dan Infrastructure as a Service (IaaS). Selain itu, terdapat empat model implementasi: \f{private cloud, community cloud, public cloud,} dan \f{hybrid cloud,} masing-masing sesuai dengan kebutuhan dan kegunaan dari organisasi yang berbeda\cite{mell2009nist}.

Kita dapat membuat sistem \cc\ sendiri dengan menggunakan Apache CloudStack. Apache CloudStack adalah perangkat lunak \oss\ yang dirancang untuk implementasi dan administrasi jaringan \vm. CloudStack berfungsi sebagai platform \cc\ Infrastructure as a Service (IaaS) yang \f{reliable}, CloudStack juga dikenal karena \f{high availability} dan skalabilitasnya. Apache CloudStack digunakan oleh berbagai penyedia layanan untuk menyediakan layanan cloud publik, CloudStack juga digunakan oleh banyak perusahaan untuk membentuk solusi \f{private cloud} atau sebagai bagian dari konfigurasi \f{hybrid cloud}\cite{cloudstackabout}.

Untuk menjalankan Apache Cloudstack diperlukan sebuah hypervisor. Hypervisor adalah komponen virtualisasi yang mengawasi dan mengelola sistem operasi \f{guest} pada sistem \f{host}\cite{scarfone2009nist}. Hypervisor mengontrol komunikasi instruksi antara sistem operasi \f{guest} dan \f{hardware} dari \f{host}\cite{scarfone2009nist}.

Pada penelitian ini \saya\ menggunakan sistem operasi Ubuntu yang berbasis Linux dan menggunakan KVM sebagai hypervisornya. Hypervisor KVM \f{(Kernel-based Virtual Machine)} adalah sebuah hypervisor berbasis kernel yang memungkinkan virtualisasi pada sistem operasi Linux\cite{whatiskvm}. KVM memanfaatkan modul kernel untuk menyediakan dukungan langsung untuk virtualisasi dengan menggunakan instruksi \f{hardware} yang mendukung teknologi virtualisasi, seperti Intel VT atau AMD-V.

Penggunaan KVM sebagai hypervisor untuk menjalankan \vm\ pada Apache Cloudstack telah menjadi praktik umum. Namun, pertanyaan mendasar muncul: apakah hypervisor ini telah dikonfigurasi secara optimal untuk menjalankan tugas yang diberikan dengan efisien? Penelitian ini bertujuan untuk menganalisis apakah konfigurasi hypervisor KVM telah diatur dengan tepat dan efisien. Pada penelitian ini, penulis akan menguji performa hypervisor dengan melakukan tugas kompresi video menggunakan Handbrake. Selanjutnya, waktu yang diperlukan untuk menyelesaikan proses kompresi ini akan diukur dan dibandingkan dalam rangka evaluasi performa.


%-----------------------------------------------------------------------------%
\section{Permasalahan}
%-----------------------------------------------------------------------------%
Pada bagian ini akan dijelaskan mengenai definisi permasalahan yang \saya~hadapi dan ingin diselesaikan serta asumsi dan batasan yang digunakan dalam menyelesaikannya.


%-----------------------------------------------------------------------------%
\subsection{Definisi Permasalahan}
%-----------------------------------------------------------------------------%
Berdasarkan latar belakang yang telah dijelaskan pada bab 1.1 bahasan mengenai rumusah yang akan dibahas pada penelitian ini adalah sebagai berikut:
\begin{enumerate}
  \item Apakah tuning hypervisor KVM pada Apache CloudStack memberikan efek yang signifikan terhadap performa sistem operasi \f{guest} yang menjalankan tugas kompresi video menggunakan Handbrake?
  \item Bagaimana konfigurasi hypervisor KVM dapat diatur secara optimal untuk meningkatkan efisiensi dalam menangani tugas kompresi video pada Apache CloudStack?
\end{enumerate}

%-----------------------------------------------------------------------------%
\subsection{Batasan Permasalahan}
%-----------------------------------------------------------------------------%
Karena berbagai kendala yang dihadapi pada skripsi ini dilakukan pembatasan masalah pada penelitian  sebagai berikut:
\begin{enumerate}
  \item Penelitian dilakukan menggunakan perangkat laptop dengan spesifikasi sebagai berikut:
  \begin{itemize}
    \item CPU: AMD A10-9620P Radeon R5 10 Compute Cores 4C+6G x64.
	\item RAM: 8 GB.
	\item Penyimpanan: 100 GB pada harddisk.
 \end{itemize}
  \item Sistem operasi yang digunakan adalah Ubuntu Server 22.04 LTS.
  \item Karena perangkat menggunakan CPU AMD, implementasi teknologi virtualisasi yang diadopsi adalah AMD-V.
  \item Parameter yang menjadi fokus analisis dalam penelitian ini adalah rata-rata perbedaan waktu kompresi video antara hypervisor default dan hypervisor yang telah dituning.
\end{enumerate}


%-----------------------------------------------------------------------------%
\section{Tujuan}
%-----------------------------------------------------------------------------%
Penelitian ini bertujuan untuk mendalami analisis performa tuning hypervisor KVM, khususnya dalam konteks penggunaan Apache CloudStack, dengan fokus pada tugas kompresi video. Tujuan utama penelitian adalah untuk mengevaluasi apakah konfigurasi hypervisor KVM telah diatur secara optimal dan efisien, serta untuk memahami dampak tuning tersebut terhadap performa sistem operasi guest. Hasil dari penelitian ini diharapkan dapat memberikan wawasan praktis bagi pengelola sistem yang menggunakan Apache CloudStack dalam lingkungan produksi, serta memberikan panduan terkait optimalisasi hypervisor KVM dalam meningkatkan kinerja layanan Cloud Computing Infrastructure as a Service (IaaS).


% %-----------------------------------------------------------------------------%
% \section{Posisi Penelitian}
% %-----------------------------------------------------------------------------%
% \todo{Posisi penelitian Anda jika dilihat secara bersamaan dengan 
% 	peneliti-peneliti lainnya. Akan lebih baik lagi jika ikut menyertakan 
% 	diagram yang menjelaskan hubungan dan keterkaitan antar 
% 	penelitian-penelitian sebelumnya}


%----------------------------------------------------------------------\hspace{0.5cm} -------%
\section{Metodologi Penelitian}
%-----------------------------------------------------------------------------%
Beberapa metodologi yang dilakukan pada penelitian ini, antara lain:
\begin{enumerate}
  \item Studi Literatur
  
  Studi literatur dalam penelitian ini dilakukan dengan melakukan pencarian, membaca, dan memahami jurnal serta sumber referensi yang terkait dengan \cc. Penelitian ini juga melibatkan eksplorasi literatur yang mendalam terkait dengan metode tuning hypervisor KVM, kondisi saat melakukan pemrosesan kompresi video, serta evaluasi keuntungan dan kerugian yang mungkin muncul akibat hasil tuning ini.
  
  \item Pengumpulan Data
  
  Pada penelitian ini, kompresi video akan dilakukan menggunakan Handbrake pada dua sistem, yaitu sistem yang menggunakan KVM default dan sistem yang menggunakan KVM yang sudah dituning. Pada masing-masing sistem, kompresi video akan dilakukan sebanyak 20 kali. Setelah itu, waktu rata-rata kompresi video pada kedua sistem akan dibandingkan dan dianalisis.
  
  \item Analisis
  
  Setelah itu, dilakukan evaluasi terhadap hasil yang diperoleh dan kemudian menyusun kesimpulan.
\end{enumerate}


%-----------------------------------------------------------------------------%
\section{Sistematika Penulisan}
%-----------------------------------------------------------------------------%
Pada penulisan skripsi ini terdiri dari 3 bab dengan pokok bahasan yang berbeda-beda untuk setiap bab.
\begin{itemize}
    \item BAB I PENDAHULUAN

    Bab I membahas tentang pendahuluan penelitian, yang meliputi latar belakang masalah, rumusan masalah, batasan masalah, tujuan penelitian, metode penelitian, dan sistematika penulisan.
    
    \item BAB II DASAR TEORI

    Bab II membahas tentang landasan teori yang digunakan untuk memahami dan menganalisis hasil penelitian.
    
    \item BAB III METODELOGI PENELITIAN
\end{itemize}