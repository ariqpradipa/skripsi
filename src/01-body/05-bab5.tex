%-----------------------------------------------------------------------------%
\chapter{\babLima}
%-----------------------------------------------------------------------------%

Berdasarkan hasil pengujian hasil konfigurasi hypervisor dari tiga skenario berbeda yang telah dilakukan, dapat disimpulkan berbagai hal:

\begin{enumerate}
	\item Flag default yang disediakan oleh hypervisor bersifat minimal, sehingga perlu dilakukan konfigurasi untuk meningkatkan performa sesuai dengan kebutuhan dari pengguna.
	\item Dari hasil pengujian konfigurasi hypervisor dengan kompresi video dengan menggunakan tools ffmpeg, kombinasi feature flag yang paling baik untuk digunakan adalah SSSE3, SSE4.1, SSE4.2, dan SSE4a dengan kecepatan kompresi video sebesar 2.4 kali lebih cepat dibandingkan dengan flag default.
	\item Dari hasil pengujian konfigurasi hypervisor dengan validasi integritas data menggunakan fungsu CRC32, penambahan feature flag sse4.2 dapat mempercepat proses validasi integritas file sebesar 7.56 kali lebih cepat dibandingkan dengan flag default.
	\item Dari hasil pengujian konfigurasi hypervisor dengan enkrispi AES, penambahan feature flag aes dapat mempercepat proses enkripsi AES sebesar 2.1 kali lebih cepat dibandingkan dengan flag default.
	\item Dari hasil pengujian konfigurasi hypervisor dengan dekripsi AES, penambahan feature flag aes dapat mempercepat proses dekripsi AES sebesar 2.68 kali lebih cepat dibandingkan dengan flag default.
	\item Dengan melakukan konfigurasi hypervisor yang tepat, performa dalam menjalankan aplikasi dapat ditingkatkan sesuai dengan kebutuhan pengguna.
	\item Untuk melakukan konfigurasi hypervisor KVM pada Virtual Machine di lingkungan Apache CloudStack dapat dilakukan dengan menggunakan virsh dan mengubah konfigurasi XML dari Virtual Machine yang diinginkan.
\end{enumerate}

%-----------------------------------------------------------------------------%
\section{Saran}
%-----------------------------------------------------------------------------%
Setelah dilakukan penelitian dan pengujian terhadap konfigurasi hypervisor untuk \vm\ di Apache Cloudstack, \saya\ menemukan beberapa sayran yang bisa menjadi pertimbangan untuk meningkatkan penelitian pada topik ini kedepannya:

\begin{itemize}
	\item Menggunakan jenis CPU yang berbeda untuk melakukan konfigurasi hypervisor, karena setiap jenis CPU memiliki karakteristik dan fungsi yang berbeda.
	\item Melakukan evaluasi pengaruh konfigurasi hypervisor terhadap kinerja aplikasi yang berbeda, misalnya, aplikasi web, basis data, atau aplikasi real-time.
	\item Memberikan contoh kasus penggunaan yang lebih kompleks, seperti penggunaan hypervisor untuk kebutuhan \textit{machine learning} atau \textit{big data} dan menganalisis efek dari konfigurasi hypervisor pada kasus tersebut.
	\item Untuk konteks penelitian konfigurasi hypervisor, dapat digunakan jenis Cloud Management Platform (CMP) yang berbeda, seperti OpenStack, ManageIQ, atau Cloudify.
	\item Membandingkan efektivitas konfigurasi hypervisor pada lingkungan virtualisasi yang berbeda, seperti (\textit{full virtualization}), paravirtualisasi, atau kontainerisasi.
	\item Menggunakan flag AVX/AVX2 untuk melakukan pengujian konfigurasi dengan kompresi video. Dikarenakan AVX merupakan flag pembaruan dari SSE.
	\item Mencari pengaruh dari konfigurasi hypervisor terhadap kinerja dari virtual machine lain di cluster yang sama.
\end{itemize}